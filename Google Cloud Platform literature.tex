

\documentclass[10pt,]{article}
\usepackage{cite}
\usepackage{zed-csp,graphicx,color}%from
\pagenumbering{roman}
\begin{document}

\begin{titlepage}
\begin{figure}[h]
  \centerline{\small MAKERERE 
  \includegraphics[width=0.1\textwidth]  {muk_log} UNIVERSITY}
\end{figure}
\centerline{COLLEGE OF COMPUTING AND INFORMATION SCIENCES\\}
\centerline{DEPARTMENT OF COMPUTER SCIENCE\\}
\centerline{SCHOOL OF COMPUTING AND INFORMATICS TECHNOLOGY\\}
\centerline{COURSE WORK :RESEARCH METHODOLOGY\\}
\paragraph*{•}
\centerline{LITERATURE GOOGLE CLOUD PLATFORM\\}
\paragraph*{•}
\centerline{prepared by:\\}
\centerline{Nakyewa Irene, 16/U/851,216000677\\}
\paragraph*{•}


\paragraph*{•}
\centerline{Supervisor: Ernest Mwebaze\\}
\centerline{ $March,7^{th},2018$\\}


\paragraph*{•}
\paragraph*{•}
  \begin{flushright}
 Google Cloud Platform ,\\
 
 \tableofcontents

  \end{flushright}
\date{\today}
\end{titlepage}

\newpage

\pagenumbering{arabic}
\section{Introduction}
Google Cloud Platform, offered by Google, is a suite of cloud computing services that runs on the same infrastructure that Google uses internally for its end-user products, such as Google Search and YouTube.\cite{s1} Alongside a set of management tools, it provides a series of modular cloud services including computing, data storage, data analytics and machine learning.\cite{s2}Registration requires a credit card gmail account information or bank account details. \cite{s3}.Google Cloud Platform is one of the most important Cloud Computing services currently available. Started in April 2008 with the release of the PaaS platform Google App Engine, it has quickly grown, adding the critical IaaS (Infrastructure as a Service) Google Compute Engine. 
\subsection{GCP Projects}
The Google Cloud Platform Console provides a web-based, graphical user interface that  can be used to manage  GCP projects and resources. When the GCP Console is used ,  a new project is created, or choose an existing project, and use the resources that you create in the context of that project. Any GCP resources that are allocated and used must belong to a project. A project can be thought of as an organizing entity for what you're building example by communicating through an internal network, subject to the regions-and-zones rules.Each GCP project has: A project name, which you provide. A project ID or GCP , project number, which GCP provides.Popular products about GCP include:App Engine – PaaS for application hosting.\cite{s15},Cloud Datastore – DBaaS providing a document-oriented database.\cite{s8}, Applications can communicate via Pub/Sub, without direct integration between the applications themselves.\cite{s5}
\section{ GCP View from users}
Users of Google Cloud Platform claim that the  App Engine is good because it it can allow them to  build, run, and scale applications without breaking a sweat \cite{s12}.The engine Works perfectly for one account. However there is need to add project work account, every time an upload is made , it redirects  to adding the work account, deletes old device policy and adds a new device policy.
\section{conclusion}
Not having true offline access to documents in Google Docs ,format is a big drawback–it would be really nice to be able to edit documents even when away from a constant Internet connection. Therefore, I would really like to see Google develop a method for offline editing like service InSync has managed to do (by converting Google Docs to local machine compatible formats). Other smaller issues include the service refusing to upload certain files and that, in my opinion, the web interface leaves something to be desired.\cite{s11}.

\bibliographystyle{IEEEtran}
\bibliography{database}   
\end{document}






